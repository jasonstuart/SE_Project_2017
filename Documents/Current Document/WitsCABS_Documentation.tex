\documentclass[a4paper,12pt]{article}
\usepackage{graphicx}
\graphicspath{{images/}}

\begin{document}
\title{WitsCABS Documentation\\Project 10\\Group 10\\}
\author{ Group Leader: 1171733 - Jason Stuart\\ \\ \underline{Front End} \\886515 - Amine Boukrout\\ 669006 - Rashad Akoodie\\\\\underline{Back End}\\1064934 - Robert Basson\\ 1171733 - Jason Stuart (GL)\\ \\}
\date{\today}
\maketitle
Github Repository Link: https://github.com/jasonstuart/SE\_Project\_2017
\newpage
\tableofcontents
\newpage
\section{Introduction}
\subsection{Purpose}
The purpose of this document is to describe what is required for the development of WITSCABS (a lift service similar to the one provided by Uber). This document is meant to convey how we have decided to develop our system and the functionality required for the first release. This will be achieved through a description of the scope of the software being designed, so that the project bounds are clearly defined, in order to prevent scope creep. The document will then for both the front and back ends describe their functional requirements.There will also be a description of the software required to be integrated with our software solution in order for it to be functional according to the requirements. 
In addition to this, a fully descriptive documentation of all programming techniques as well as software engineering techniques that we as a group will follow during development will be included. These techniques will be substantiated in order to provide a clear systematic approach to the complete design of our project. Furthermore, all resources consulted will be included as well.
\subsection{Problem Statement}
WITS students often require transport to get them from place to place, it could be to and from a place of residence, shopping centers, places to support wits sports, internet cafes etc. Wits would like to implement a transport system similar to Uber that Wits students are able to use as a means of travel.

\subsection{Project Objective}
The objective of this project is not only to learn and make use of specific design procedures and team management skills, but also to further our skills in development of software using specific techniques that will be used in industry. Further, the objective is to have a working prototype of the WitsCABS project with the primary functionality operating by the end of the time allocated by the client. The project team would also like to gain experience in project development and management as it will be useful in our careers.
\newpage
\subsection{Scope}
Our core objective is to design, develop and implement a software system which will be used to manage a fleet of taxi-cabs for a company (as per the project brief). Our company, “WitsCABS Company”, will be independently run and will employ our own staff/drivers. Vehicles however may either be owned by WitsCABS Company or by the drivers. The service will run based on “service zones” within the city of operation i.e. the city will be split into various sectors which will allow for more efficient ride time. Each driver will be equipped with a smart-phone embedded with GPS, maps and a navigational tracking system. All vehicles being dispatched will be on instruction from the 24-hour manned Dispatching Control Centre (DCC) which will be in direct contact with clients via calls to the centre.
\\\\
The front-end of our system will comprise mainly of an android application used by the drivers which will be integrated with a GPS and mapping service such as Google maps to optimize routes taken as well as save time in the case of human error. To achieve this, we will be primarily using Java as a programming language as well as some JSON, HTTP and Javascript. The front-end of our mobile android application will consist of the visual representation of the map being used as well as any turning signals or warning, messages etc. needing to be displayed on the screen, a login screen for the drivers where they would enter their credentials, a waiting screen for when they are awaiting a client. Furthermore, the drivers should communicate with the DCC. For this, a web-based application and/or application will need to be created for the DCC where they can view the entire city of operations. The front-end of this “application” would comprise of a form that will allow the DCC to capture customer details. It will then send these details to the back-end which will then determine which cab to assign.\\ \\
The Back-end of our system will make use of Java as well as JSON. This will consist of storing all the driver’s information (Names, license number, car registration, age etc.), details related to each trip the driver makes as well as any necessary information about the passenger transported. It will also be integral to integrate the maps into finding which driver is best suited for the next pickup requested based on some sort of efficiency algorithm which we will design. This information would most probably need to be stored in a database and communicate with our mobile application using JSON. \\\\
\includegraphics[scale=0.7]{user_flow}
\pagebreak
\subsection{Overview}
The document will be broken up into multiple sections. As already mentioned before in the document, we have given an introduction as to why this document has been typed up, the objectives to be met, and the problem statement that describes the problem at hand. The scope of what the project is all about and a basic description of the required features needed to be implemented is also discussed above. We also discuss the relevant stakeholders next. In section 2, an analysis of how the project will be conducted is discussed. This includes a discussion of system architecture types, software required for the project, as well as how the team will be managed and handled during development. In section 3, a list of formal required functions will be provided for both front-end software as well as the back-end. Section 4 will then go into the design documentation, such as Database design, e.t.c. Section 5 will then contain relevant documentation related to Sprint Planning as well as Sprint Retrospectives for the SCRUM Methodology. Then the document will discuss how to evaluate the software for the project owner's use in Section 6. Then in conclusion a list of responsibilities and credentials are provided in section 7 and a project Sign off is completed in section 8.
\newpage
\subsection{Stakeholders}
The following people have an interest in this project:
\subsubsection{Users}
The future users of this project, being students that require transport within service areas, will be keen to use this software as transport is still a major issue in South Africa. This often leads to people not getting to their destination on time, a key factor for many. Hence these users will be keen to know how this software turns out.
\subsubsection{Developers}
Apart from the developers having to do this for their job, they too will be interested in making sure this project flows and is completed by the end of it, as it will further develop their skills as well as could help them further once the project is completed. 
\subsubsection{Project Management}
Project Managers, the ones who handle the project at an administration level also have similar goals to developers.
\subsubsection{Project Owners}
These are the people who initiated the idea of this project and have since contacted us to implement such a project. This will provide good gains to them as it will improve the reputation of their company as they try to solve a common problem in South Africa.

\pagebreak
\section{Analysis}
\subsection{Team Administration}
\subsubsection{Team Management}
Our team has decided to take the Agile SCRUM method as our SDLC (software development life cycle). We chose this approach based on multiple reasons. Due to the short time constraint given to us by the client, a rapid development approach is needed, one that is satisfied by Agile development.\\\\ We also require quite a bit of feedback from the client, hence this iterative approach is best as it allows us to be in constant communication with the client, with small changes in between, allowing us to quickly change anything the client is unhappy with. Another advantage of SCRUM specifically, is that we will constantly be meeting up each day either in person, or by Google Hangouts to check up on the group and what each member has done over the last day, allowing us to keep a high morale and high sense of worth in the team, even though the time requirement is strictly short. Since our team is highly motivated already and have experience in development, they can be trusted to handle their tasks efficiently, another benefit of the SCRUM process, where each developer has more responsibility than in other methodologies. Also, since our team is relatively small, SCRUM will also work, as there will not be as much time required to structure and organize the team as a whole, increasing productivity. Also, since SCRUM focuses more on the actual development and less on the documentation, since we are going to be showing each change to the client, it benefits the productivity even more.
\\\\We have also considered the disadvantages of SCRUM. We understand that there is a higher chance of Scope creep with Scrum, but this is why we have already defined the scope of the project in the beginning of this document, to define clear boundaries to the project, to prevent this from happening. We have also been very clear with our functional requirements needed for a successful project, to prevent inaccurate measurement of time estimations.\\\\ Furthermore, SCRUM has the possibility of failure, if at least 1 member loses interest or focus. This is why we have nominated a clear leader to take charge and to keep the team motivated, regardless of what may come. If a team member were to leave, this would also have a large inverse effect on the project, however this will not happen, as each member is fully committed to this project, as this project results in marks required to graduate. \\\\Finally, it may be hard to quantify quality as the software is rapidly evolving, however since we do not have a quality control team, we can work on quality between each developer to keep the project from slipping quality wise.
\subsubsection{Team Meetings}
The team will conduct the project as follows:
\begin{itemize}
\setlength\itemsep{0em}
\item Each team member will be required to be at each sprint planning meeting every week on Friday, from the start of the project. This meeting will be organized by the Group Leader, i.e. the scrum master.
\item Each member will be required to be available for the daily stand up meeting, preferably in person, otherwise via Google Hangouts. Each member should be able to describe to the team what they have worked on over the past 24 hours. They will also describe any problems they face and discuss any possible solutions to the problems. Thereafter each member will describe what they will work on for the next 24 hours. 
\item The scrum master will ensure that the project is moving at a reasonable pace and that all team members are motivated at all times.
\end{itemize}
\subsubsection{Responsibilities of Team}
Our current student assignment is as follows. Rashad and Amine will handle the mobile app. Robert will handle the front-end call center app. Jason will handle the back-end and database. As the project moves forward, this can change, depending on who needs what help at each time. Each member is responsible for providing constant communication and feedback on their portions as well as suggestions on others work. Jason will schedule any meetings required to advance the project in any way.
\newpage
\subsection{System Architecture Analysis}
After thorough analysis of existing architectures and analyzing the functional requirements listed later in this document, the team has decided to make use of a 2 part Client - 1 part Server architecture. The first client software will be the mobile app that allows the drivers of the WITSCABS service to receive communication from the system as to who the driver should pick up, etc. \\\\The other client software will be run on a desktop in the call center, so that as a client calls in, the operator can input all the details necessary for the system to compute the best driver to use. The server will be the bridge between these two client software services, where the server determines the best driver for the job that arrives and pushes the required information through to the driver on his mobile app.\\\\ All this communication will be facilitated through an Internet connection, so when designing the system, it must take into consideration that an internet connection might not always be available for the drivers. There is no need to have any further server or client interfaces thereafter.
\newpage
\subsection{Analysis of Front-end System (Mobile App)}
We have decided to make use of an android application as the primary user-accessible interface for our project. The reason for choosing a mobile application is quite obvious in our case as the drivers who will be using the application have to be in their vehicle at the time of using the application as well as moving in the vehicle while using it. This automatically squashes any idea of a desktop application or web based application as the accessibility would be a major issue. The possible downside to this would be that each driver requires a cellphone/tablet which will support the software WITSCabs as well as a mobile data connection. \\\\
The call centre will be using a desktop application in conjunction with the android application although the users will not be registered to the database but instead will have access to the database in a read-only capacity. This will allow the call-centre to access client details as well as driver’s details which will in turn allow them to run statistics checks as well as be prepared in the case of an emergency. For this, it is clearly most beneficial to use a desktop application preferably with an internet connection as immediate run-time statistics as well as driver locations etc. are imperative.\\\\
For this purpose we have chosen to develop our software as an android application as it is the most common operating system friendly platform for a mobile application. Linking with this, it has been decided as a group that a predominant bulk of the code used should be Java as it is the most commonly used programming language for Android applications as well as the most familiar between each member of the group. Adding to this, we have decided to use Android Studio as our IDE for our app development as it is popular, free to students and boasts many high-end features which will benefit the development of our project.
\newpage
\subsection{Analysis of Front-End System (Desktop App)}
When a user wants to get a lift they have to call in and speak to a call center agent. The call center agent will then get the relevant information from the user and capture it in a desktop application. This application will communicate with a server which will capture the information in a database. The server then sends a notification to the mobile application so that a driver can pick up the user.\\\\
We have chosen to use a desktop application over a web application. The main reason for this is responsiveness. Web applications can become unresponsive if the internet connection is heavily used or if there are connectivity issues such as a damaged line, desktop applications have a more reliable response time as response time will be limited by the CPU. Desktop applications are also more secure than web applications which will be important as we store user information on a database.\\\\
The language chosen to develop the desktop app is Java. Java is platform independent so it will run no matter what operating system is on the call center computers. Java is an easy to learn object oriented language, this helps us to create reusable code. Another benefit of Java being easy to learn is its popularity, this means that should any problems arise it won't be difficult to find someone to fix these problems.
\newpage
\subsection{Analysis of Back-End System}
The back-end will consist of two parts. The server application and the database.
Our database will be developed using the DBMS MySQL. MySQL is very commonly used and a simple DBMS so any issues with the database can be dealt with quickly and easily. As MySQL is very easy it only requires basic SQL statements in order to interact with the database. We have also chosen MySQL as it is free which helps reduce the cost needed to develop the full system. MySQL is also able to handle large amounts of data which is well suited for a popular application. The DBMS also includes options to add security which can help protect the data in the database from intruders.
\\\\
The development of the back-end is important even though the user does not directly interact with it. We need a means to store user information and assign drivers. This back-end actually serves as the lifeline for the system. Hence this will also be developed with Java, since all the members of the team are most familiar with the language and also understand how to implement networking to the highest efficiency possible using Java.
\subsection{Additional Software API’s Required}
Due to the nature of the project, the implementation of the software application will require external software and most probably certain add-ons from the supporting software. One of the main supporting software that is required to actually implement the application is Google maps along with its various API’s. This is required since the application has to be integrate with Google maps in order to provide a GPS service to guide the driver to the location of the customer. \\\\
A type of a database management system (DBMS) will be required in order to keep relevant information about the operations of the business. Such information would include login details, customer details, driver details, trip details, etc. Most probably the DBMS that will be used is MySQL. It is worthwhile to note that as the project progresses additional software requirements may arise, and in this regard the client will be notified.

\subsection{Analysis of Inputs and Outputs}
\subsubsection{Inputs}
The first type of input would be a phone call from a customer providing information such as the pickup location, customer name, and contact number. The dispatching control center (DCC) agent will enter the information provided by the customer into a database by using an interface that will run on a localised desktop at the DCC. Once the information is stored on the database, the system will allocate a customer to a driver based on whether or not a driver is available or not (by using an algorithm designed and agreed upon by the developers). The driver will then be provided with the GPS information of the customer which will be inputted into the embedded GPS on the driver's smartphone.

\subsubsection{Outputs}
Once the driver has been allocated, a notification will be sent firstly to the driver via the designed app to notify him/her about the customer's assignment to that particular driver and basic information, such as the customer's name and contact number, along with the physical address or GPS coordinates that the customer provided (this can be displayed the driver's app as link which could launch the designed application for the company or the GPS app embedded in the smart phone). On the customer's side, he/she will receive a SMS when the driver is in a predefined range from the address provided by the customer.

\pagebreak
\section{Functional Requirements}
\subsection{Front-End Mobile App}
\begin{itemize}
\setlength\itemsep{0em}
\item Send Status updates to the server (updates server configuration to know whether this driver is available or not)
\item Send location updates so the server knows where you are when status is marked available.
\item Be able to receive passenger information that will help the driver find the passenger.
\item Find the quickest route to the destination.
\item Update Driver details
\item Register Driver onto the system.
\end{itemize}
\subsection{Front-End Desktop Application}
\begin{itemize}
\setlength\itemsep{0em}
\item Create new passenger
\item Insert passenger into database via the server
\item Edit record
\item Push and save information to database
\end{itemize}
\subsection{Back-End Server}
\begin{itemize}
\setlength\itemsep{0em}
\item Accept incoming connections from desktop applications, which contains details related to the passenger in need of transport.
\item Analyze which service zone the passenger is located in.
\item Run an algorithm to find the closest driver ready to pick up a passenger
\item Send a push message to this specific driver with the details of the passenger to fetch as well as their destination.
\item Allow drivers to mark if available or not via a status change
\item Be notified when a driver is close to pick up point via SMS.
\item Be notified when passenger is there at their destination, to automatically mark drivers as complete, i.e. the driver is transitioning back to their house or service taxi rank. Thereafter the driver can mark they are ready for another job.
\end{itemize}
\pagebreak

\section{Design Documentation}
\subsection{Database Design}
\includegraphics[scale=0.5]{Database_Diagram}
The approach taken here is to achieve as minimalistic approach, storing the least amount of information possible. Hence we just have 3 tables, one for the customer and their details, one for the driver and their details and a bridge entity to match the two together. To simplify things, we decided that if a driver is part of a rank, their home address details become the ranks address. This is due to the fact that we only need to know where they sit waiting for a new drive. It will also help improve the efficiency of the algorithm that assigns the customer to a driver. We also made it that the customers do not have to each time log in, they just call and create a new customer each time for convenience.

\subsection{Server Design}
\subsubsection{Class Diagram}
\includegraphics[scale=0.5]{Server_UML}
When designing the server, we as a group decided that it would be best to make sure efficiency was our main priority, especially due to the networking specifics required. Hence we went for a minimalist approach and kept all networking as simple as possible. We also made use of threads to improve response time when multiple queries came in at once, as each thread could handle a request at the same time. Further, we decided to separate the code into multiple classes and methods to make the code more understandable and easier to read, as well as make sure that there is upgradability and maintainability with the code, for further development. The image above describes the class diagram that links all the classes together.
\newpage
\subsubsection{Sequence Diagrams}
\includegraphics[scale=0.5]{CreateSequenceDiagram}
The above diagram shows the general procedure that is followed when either a new customer or driver is created. The steps that the applications take are outlined in this diagram above.\\
\includegraphics[scale=0.5]{GetDataSequenceDiagram}
Similarly, the above diagram shows how the front end apps query the server for any data they need. They again, follow the same general strategy as outlined in the sequence diagram.

\subsubsection{State Machine Transition Diagram}
\includegraphics[scale=0.5]{statemachinediagram}
\\The above diagram shows how the state of a drive progresses as events occur. When the customer is newly created from the DCC or call center, they are ``assigned'' to a driver, but the driver does not know this, until the app refreshes with the server. Once the driver has the refreshed information, i.e. the drive has been transferred to a driver, then the drive is ``InProgress''. Finally, once the drive is ``completed'', it is marked as so, and archived.
\newpage
\subsubsection{Driver Assignment Algorithm}
Apart from the previously mentioned and discussed design diagrams, it is critical to discuss the design of the algorithm that chooses which driver to dispatch to each customer. The server gets a resultSet from the database of all drivers ready for a drive in the service zone related to the customers starting position. It then, using the distanceMatrix API provided by Google calculates which driver will get to the customer first in terms of time taken (including traffic delays e.t.c.) and assigns that driver to go fetch the customer. If the closest drivers are located at the Service Rank, the algorithm picks the driver that marked their status as available first at the service rank. It also then marks the driver status as unavailable and the server then waits for the driver to get the customer details.
\newpage
\subsection{Call Center Design}
\includegraphics[scale=0.5]{CallCenterSequence}\\
The diagram shows the process that occurs when a user calls the call-center and the call-center agent captures their details on the form. The steps that need to occur are outlined above.
\newpage
\subsection{Android App Design}
\includegraphics[scale=0.5]{AndroidAppUML}\\
As any other android applications, the application consists of activities which are standalone and
completely independent from one another. (To clarify terminology, an activity is a screen that displays on
the android device in use.) The application also consists of several auxiliary classes which assist in the
functionalities of the main activities (i.e. the activities).
The WitsCabs android application has five activities in total, wherein three out of the five activities are the
core activities of the application. Each of the five activities are standalone as stated previously, and thus
each of the five activities constitute to different and specific functionalities of the application. Each of
the activities are linked to a single .xml file which helps to display the visual aspect of the activity (the
GUI) on the device.\\The different (main) activities along with their functionalities are described below:
\begin{itemize}
\item StartUpActivity: this activity is linked to the startupactivity.xml. This specific activity is launched
when the application is launched. The StartUpActivity displays an image along with text that
reads ``Welcome to WitsCabs!'', and automatically times out after 7 seconds to start up the next
activity which is the login page for the application.
\item loginPage: this activity is linked to the loginlayout.xml and is launched automatically after the
StartUpActivity. This activity allows the user (i.e. the driver) to login to the application by
inserting his/her username and password in the provided text boxes followed by clicking the
login button, and also provides a button which the user can click to register as a driver at
WitsCabs. Depending on which button is clicked the application will lead to a different activity. If
the login button is clicked, the application will then lead to the dashboard of the application and
if the register button is clicked, the application will lead to the register activity.
\item signUpPage: this activity is linked to the signuplayout.xml and is launched when the ``Register''
button is clicked in the loginPage activity. This activity consists of several text boxes which
allows the user to insert his/her information into. Note that these text boxes have ``hints'' which
indicates to the user what information is required to be entered by the user at each text box. At
the bottom of the application there is a button named ``Register'' to officially register the user, and once
clicked and if successful the application will lead back to the loginPage activity and if
unsuccessful the application will stay on the same activity and display a not registered message
via a toast.
\item DashBoard: this activity is linked to the dashboardlayout.xml and is launched when the user is
successfully logged in, as well as this activity is the ``main activity'' after logging in. In this activity
is where the user can change his availability status, there is an analogue clock that displays the
time, a text view section which displays the details of the passenger that has been allocated to
the specific driver, and there is a ``Go To Maps'' button which once clicked the application leads
to the MapsActivity.
\end{itemize}
\underline{Note:} all of the above activities extends from the Activity superclass.

\begin{itemize}
\item MapsActivity: this activity is linked to the activity\_maps.xml and is launched when the ``Go To
Maps'' button is clicked in the dashboard. The main purpose of the activity is to give directions
to the driver to get to the passenger, and to get the passenger to his/her destination. This
activity makes use of the google maps API and google directions API, along with the auxiliary
predefined modules by google. These modules include DirectionFinder determines the route of
the trip required, DirectionFinderListener which listens when directions are requested, Duration
calculates and stores the duration of the current trip, Route stores the route.
\end{itemize}
Auxiliary classes:
\begin{itemize}
\item AsyncClassSignUp and AsyncClassLogin: initiates the communication with the back-end server in
order to send the required registration and login information respectively to the server, and
receives a response from the server. Based on the response, certain decisions are made in the
signUpPage and loginPage respectively. Please note that the IP address when declaring the
Socket has to be altered based on the machine that the back-end server is being run on.
\item JSON\_Handler: converts an array of strings to a JSON string in order to send data to the server
using a JSON string format. This class was created by Jason for use in the server, but we made use of it in the android app as well.
\end{itemize}
\underline{Important Notes:}\\ The actual application is not fully as required due time constraints and issues with the google
API, however basic functionality of the app is operating.
\newpage
\section{SCRUM Documentation}
Note: Each standard sprint we implemented has a length of 1 week. It should also be noted that each day at 19:00 the team would do a daily stand-up meeting via Google Hangouts wherein each member would describe what they have done, the problems they faced, as well as get any answers related to their concerns if they needed to do so.
\subsection{Sprint 1: 7 September 2017}

\subsubsection{Sprint Planning Documentation}
\begin{tabular}{|p{9.5cm}|p{3.5cm}|}
\hline
Task & Assigned \\ \hline
Create Database Structure & Jason \\ \hline
Implement Server Socket for backend & Jason \\ \hline
Implement Threading for server & Jason \\ \hline
Implement Call Center App & Robert \\ \hline
Link Call Center and Backend Together & Robert And Jason \\ \hline
Construct Android Screens & Amine and Rashad \\ \hline
Analyze Google API for use & Jason \\ \hline
\end{tabular}

\subsubsection{Sprint Retrospective}
\underline{Overview}\\
All Tasks Completed Successfully Except for the Android Screens. Amine and Rashad encountered a strange bug that would not allow them to compile their code, hence they have been held up and have not met the deadline. Their team has gone to get assistance with the bug from staff who are knowledgeable with android development. \\ Robert Successfully completed the Call Center Screens within 2 days, but the holdup was implementing the networking code as we had issues with the proxy blocking our connection. We found a workaround and now have a working connection. \\ Jason installed and created the mysql database as discussed in the planning stages of the project. He also implemented the server networking code and Threading code as well. He also helped Robert connect the applications together. He further went on to analyze the google api and even implemented it into the server code ahead of schedule.\\\\\\\\
\underline{Negatives}\\
Currently the android team is behind schedule due to the bug. Hence it is critical that they can recover from this situation and catch up. This means they will have to work extra hard this next week. Also the networking methodology was not clearly described by Jason until later on in the week, as he was still implementing the code, hence the networking of Robert's code could not be implemented yet. \\\
\underline{Positives}\\
Jason managed to get ahead of schedule and implement slightly more than what was required for the week. This puts us in a relatively good spot as more time can go into making sure the server is working to its optimal pace. Communication within the team on each day was also excellent, with each member checking in on Google hangouts for the daily stand-up meeting. The Android team also immediately let all involved know that they had run into this IDE bug. 

\newpage

\subsection{Sprint 2: 15 September 2017}
\subsubsection{Sprint Planning Document}
\begin{tabular}{|p{9.5cm}|p{3.5cm}|}
\hline
Task & Assigned \\ \hline
Implement all Server functionality required for both front-ends & Jason \\ \hline
Finish Android Screens & Amine and Rashad \\ \hline
Implement networking code for android app to server & Amine and Rashad \\ \hline
Implement functionality of basic functions for android prototype & Amine and Rashad \\ \hline
Implement Google Maps API into android app & Amine and Rashad \\ \hline
Touch up beauty aspects of Call center app & Robert \\ \hline
Test Connection and functionality between server and call center & Jason and Robert \\ \hline
Begin basic testing of server functionality and find bugs & Jason \\ \hline

\end{tabular}

\subsubsection{Sprint Retrospective}
\underline{Overview}\\
This sprint went much better than the previous one. While the android team is still behind, they feel that they can complete the job by the end of sprint 3. They managed to add the required functionality and implement the Google maps api into the app. They also managed to complete the android screens. They now just need to add the networking code. Robert finished commenting and tidying up the code for the call-center app. Jason did bug testing excluding the android side of things on the server. \\
\underline{Negatives}\\
Once again, the android team encountered some strange bugs with the IDE while developing the app, this slowed them down and in the end they could not finish their networking portion of the code. They are however more than ever determined to sidestep these problems to the best of their ability and make sure that all can be done as quickly and efficiently as possible.\\
\underline{Positives}\\
Our group has begun to work more with each other and less on their own, speeding up the development process. Furthermore, it has become more clear to each member exactly what is required of them. Every member is motivated and positive they can finish all content in the next sprint.

\subsection{Sprint 3: 22 September 2017}
\subsubsection{Sprint Planning Document}
\begin{tabular}{|p{9.5cm}|p{3.5cm}|}
\hline
Task & Assigned \\ \hline
Implement networking code for android app to server & Amine and Rashad \\ \hline
Bug test server code with android app & Whole Team \\ \hline
Advanced testing of server functionality and find bugs & Jason \\ \hline
Begin overall test cases and find bugs in all software & Whole Team \\ \hline
\end{tabular}

\subsubsection{Sprint Retrospective}
\underline{Overview}\\
The team is pleased to say that the networking code between the android app and server now successfully runs for some functions, meaning that the app is now in a demonstrable state. This took a lot of effort to program as we were constantly facing new challenges, but through perseverance, a strong quality of the team, it has been done. The main focus of this week however was bug testing, which the team did an admirable job finding these small but significant bugs in the software code. Testing also occurred which was a huge success. It should be noted that not all the android functions were implemented, but that at least the core networking has been implemented.\\
\underline{Negatives}\\
The team could not identify any Negative aspects relating to this week. All went smoothly. The only thing is that the group wished they could have a fully functional android app with all functions implemented.\\
\underline{Positives}\\
The team has been putting in extra hours behind the scenes to make sure that the project deadline is met, even though these hours are not officially scholar hours. This shows that the team is dedicated and motivated to give their all in making sure this project completes successfully. 

\subsection{Mini Sprint 4: 29 September 2017 - 1 October 2017}
During this time period, a major focus was made on neatening the document up into a final document that is legible and understandable. It should be noted that the document was worked on at all points during the development stages and evidence of this is provided on Github in the repository by analyzing commits and their contents.

\newpage
\section{Evaluation}
Note: The Github Repository Link is: https://github.com/jasonstuart/SE\_Project\_2017
\subsection{Compiling And Running Code}
\subsubsection{Desktop Application}
The Desktop Application to be used in the call center will be a Java program (as previously mentioned). So in order to compile it an IDE that supports Java will be required eg Eclipse or Netbeans. For the sake of this document it is assumed that Netbeans will be used. Further it is required that you have the latest JDK and JRE software installed from Oracle.\\
Netbeans has a built in compiler that is easy to use. On the ribbon at the top of screen there are two hammers, one hammer has a brush. These hammers are used to call ``build'' and ``clean and build''. Clicking either one of them will compile the code. The difference between the two is that clean and build completely recompiles the code, deleting former versions.\\
After compiling a .jar file should be present in the dist folder of the Java project. This .jar  file can be transferred to computers in the call center and will allow the application to run without the need for an IDE. It will require the use of the command prompt.\\To run the code using an IDE there are multiple options, you can right click the main class and select run file, or click the green play button in the top ribbon. Either of these options will run the code if the code has been compiled or compile and then run the code. To run from the command line without the IDE installed, navigate to the file in the command prompt using the cd command and then type: java -jar CallCenter.jar
\newpage
\subsubsection{Server}
Similarly to the Call Center Desktop Application, the server has also been constructed using Java as the programming language in order to more easily maintain compatibility with all the other programs. So in order to compile the code, you can follow a similar approach to 6.1.1 for the desktop application. Note it is required that you have installed the latest java JDK as well as JRE software. There is however an extra step required to setup the database to work with the server. One needs to install mySQL and install it to the system using the default settings. When setting up user details, set the root password to ``jason''. Keep the port number the default. Furthermore, make sure to restore the current database from the repository under the folder heading MySQL Server (in the server project folder) to the server running on your machine using the command: mysql -u root -p WitsCABS < WitsCABS.sql where this command is run in the same directory as the sql backup is located. Finally, to make sure that you can receive connections from client programs, make sure to give them your current computer ip address to input into their program code for testing. This can be done using your command prompt by typing ipconfig. Note: it is critical to be on a network other than the Wits network as the proxy service and network policies interfere with the correct running of the server. Remember to also input this ip address into both the android and call-center app, or you will not be able to successfully link each piece of software together. No ip address or port number configuration is needed for the database if the instructions above have been followed.
To run the server, follow the similar setup as for the desktop application, i.e. navigate to the .jar file in the command prompt using cd and then type: java -jar WitsCABS\_Backend.jar

\newpage
\subsubsection{Android Application}
The Android application has been developed by using Android Studio as our IDE with Java as a programming language. Java has been chosen as it is the most familiar programming language between the group as well as the most used among android developers. Before installing the .apk file, one has to install the latest version of the Java JDK (1.8) and the android SDK (API 23). In order to install the app on a mobile device,  one would be required to install Android Studio. We will assume that we're using Android Studio in this description. The first thing to do would to navigate to the project structure. In the project structure, choose the JDK and SDK you downloaded on your system. 
When using a mobile device to run the application, make sure that developer options are enabled. Do this by clicking 10 times on the build number under my device info in the settings of the android device. Once this is completed, enable USB Debugging in the developer options. You are now ready to install the WitsCabs app. Simply click on the ``Run'' tab on the top of the screen and choose ``Run Application'' with the mobile device connected to the pc running Android Studio via a USB cable. When prompted to choose which device to run the application on, choose the name of the android device being used and the application .apk file will be installed on the device. Once this is completed, all that is needed is a Wi-Fi or cellular connection on the device for the app to work. Also, ensure that once the app is installed on the device, the user should give the app location permissions to the app explicitly. This can be done by going into the settings of the device, navigate to ``Applications'', find the WitsCABS app and click on it, under permissions ensure that location is active by sliding the bar to ``On''. For testing purposes, use a Wi-Fi connection that is being used by the Server application. Note: once installed, the application can be run from the android device without having to install it again.



\pagebreak
\section{Credentials}
\subsection{The Credentials}
\begin{itemize}
\setlength\itemsep{0em}
\item Purpose by Robert, Rashad, and Jason (Section 1.1)
\item Problem Statement by Robert (Section 1.2)
\item Project Objective by Jason (Section 1.3)
\item Scope by Amine (Section 1.4)
\item Image Process Diagram by Robert
\item Overview by Jason (Section 1.5)
\item Stakeholders by Jason (Section 1.6)
\item Team Management by Jason (Section 2.1.1)
\item Team Meetings by Jason (Section 2.1.2)
\item Responsibilities of team by Jason (Section 2.1.3)
\item System Architecture Analysis by Jason (Section 2.2)
\item Analysis of Front-end System (Mobile App) by Rashad (Section 2.3)
\item Analysis of Front-End System (Desktop App) by Robert (Section 2.4)
\item Analysis of Back-End System by Robert (Section 2.5)
\item Additional Software API’s Required by Amine (Section 2.6)
\item Analysis of Inputs and outputs by whole team (Section 2.7)
\item Front-End Mobile App to be handled by Amine and Rashad (Section 3.1)
\item Front-End Desktop Application to be handled by Robert (Section 3.2)
\item Back-End Server to be handled by Jason (Section 3.3)
\item Database Design by Jason (Section 4.1)
\item Server Design by Jason (Section 4.2)
\item Call Center Design by Robert (Section 4.3)
\item Android App Design by Amine (Section 4.4)
\item SCRUM Documentation by Jason as Group Leader
\item Evaluation, compiling and running Desktop Application by Robert (Section 6.1.1)
\item Evaluation, compiling and running Server and Database by Jason (Section 6.1.2)
\item Evaluation, compiling and running Android Application by Rashad (Section 6.1.3)

\section{Project Sign Off}
The following signatures related to each specific person below hereby declares that said person is happy with the Documentation, as well as the Code that has been created for the submission and that each person is happy with the amount of contribution they gave to this project (which can be seen both by the Credentials entry above in the documentation as well as checking the commit version history for analysis of coding contribution) which can be viewed by the lecturers and that said persons are ready to submit the project in its current format.

Robert Basson: \includegraphics[scale=0.12]{Robert_Signature}

Rashad Akoodie: \includegraphics[scale=0.08]{Rashad_Signature}

Amine Boukrout: \includegraphics[scale=0.05]{Amine_Signature}

Jason Stuart (Group Leader): \includegraphics[scale=0.05]{Jason_Signature}

\end{itemize}
\end{document}